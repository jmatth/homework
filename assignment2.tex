\documentclass[12pt]{article}
\usepackage[margin=0.75in]{geometry}
\usepackage{indentfirst}
\usepackage{listings}
\usepackage{amssymb}
\usepackage{amsmath}

\title{\bf Algorithms Homework 2}
\author{Kaitlin Poskaitis, Joshua Matthews, Russ Frank, and Matt Robinson}
\date{}

\begin{document}

\maketitle

\begin{enumerate}

\item

  $a = 2\\
  b = 2\\
  c = 1\\
  d = 1\\
  f(n) = \log n\\
  f(b) = \log 2 = 1\\
  \Theta (n^{\log_b a)}\\
  \Theta (n^{\log_2 2)}\\
  \Theta (n)\\$

\item Question 2

  I don't know.

\item

  \begin{enumerate}
    \item \textbf{Lemma.} Perfectly balanced BSF can be searched in $O(\log n)$
    At worst, one subtree is twice as long, so $2O(\log n)$.

    \item $n \log n$ to build tree ($n$ searches)

    \item $\frac{n - 1}{2} + 1$

    \item $\frac{n\log n}{\frac{n - 1}{2} + 1} + \frac{\log n}{\frac{n - 1}{2}}$
    
    \item $\frac{n\log n}{\frac{n - 1}{2} + 1} + \frac{\log n}{\frac{n - 1}{2}}$\\
    $\lim_{n \to \infty} \frac{n\log n}{\frac{n - 1}{2} + 1} + \lim_{n \to \infty} \frac{\log n}{\frac{n - 1}{2}}$\\
    $\lim_{n \to \infty} \frac{\log n + \frac{1}{\ln 2}}{\frac{1}{2}} + \lim_{n \to \infty} \frac{\frac{1}{n\ln 2}}{\frac{1}{2}}$\\
    $\lim_{n \to \infty} \frac{\log n + \frac{1}{\ln 2}}{\frac{1}{2}} + 0$\\
    $O(\log n)$
  \end{enumerate}

\item 

  We'll start by proving the following lemma; we'll use this lemma to
  inductively prove the given statement.

  \textbf{Lemma.} The smallest value in any column is greater than or equal to
  the smallest value in the previous column.\\
  \textbf{Proof.} Consider the following matrix:

  $\begin{matrix}
    a & b & c \\
    d & e & f \\
    g & h & i
  \end{matrix}$

  This matrix has not had its columns sorted yet. So, we'll assume $e$ is the
  smallest element in its column; therefore $e \ge d$ since the rows are sorted.

  We want to show that $e$ is greater than or equal to the smallest element of the first
  column, whatever that may be. Well, if $d$ is the smallest element then this
  is trivially true, by our assumption. If there is an element in the first
  column which is $\le d$, then clearly it is also smaller than $\le e$;
  i.e.\ for an arbitrary element $n$, $n \le d \le e$; so $e$ is still larger 
  than or equal to the smallest element of the first column.

  So, now we'll prove what was asked in the question.

  \textbf{Theorem.} If we have a 2D array with rows in sorted order, the rows
  will remain sorted after sorting the columns as well. \\
  \textbf{Proof.} After sorting, the top row of the array will contain the
  smallest element of each column. So, by our lemma, this row must be sorted.

  Eliminate that row from the array. Now, the new top row contains the new
  smallest element of that column. By our lemma, again, that row must be
  sorted. We can continue eliminiating the top row and this process will
  continue; showing that all rows are still sorted. $\blacksquare$

\item

  Since a hamiltonian cycle is simply a hamiltonian path which is also a
  cycle, we can simply run each pair of neighbors through the $HamP$ routine.
  There will be, at maximum, $n^2$ neighbors, where $n$ is the number of nodes;
  so, this whole operation will take polynomial time.

\item
  
  Well, we begin by running the $HamP$ routine on each pair of nodes. 
  Eventually we'll find a pair of nodes for which the routine returns true.
  This will take at most $n^2$ times the runtime of $HamP$ time, which is still
  polynomial.

  Then, once we have two nodes $s$ and $t$ for which this works, we remove $s$
  and run the subroutine on $t$ with each of $s$'s former neighbors. By 
  repeating this process every time the subroutine succeeds, we can determine
  the path.

  Again, since there are at most $n^2$ edges in the graph, we'll need to run the
  $HamP$ routine a maximum of $n^2$ times. So, the runtime of this entire
  operation is still polynomial.

\end{enumerate}

\end{document}
