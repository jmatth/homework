\documentclass[12pt]{article}
\usepackage[margin=0.75in]{geometry}
\usepackage{indentfirst}
\usepackage{listings}
\usepackage{amssymb}

\title{\bf Algorithms Homework 4}
\author{Kaitlin Poskaitis, Joshua Matthews, Russ Frank, and Matt Robinson}
\date{}

\begin{document}

\maketitle

\begin{enumerate}
%Question 1
\item The Longest Common Prefix problem is defined as follows:

    {\bf Preprocess}: $D$ is a set of $n$ strings, each of which is
        length $m$.

    {\bf Queries}: $LCP(i,j)$ returns $k$ if $S_i[1...k] = S_j[1...k]$
        but $S_i[1...k+1] \neq S_j[1...k+1]$

Give an algorithm for this where the query time is constant.

Preprocess implementation:
\begin{itemize}
    \item Build a prefix tree of all strings in $D$. $O(nm)$
    \item Create a hash map mapping each word it its leaf in the prefix tree.
        $O(n)$
    \item Preprocess for $LCA$ for nodes in hash map. $O(n)$
    \item Total runtime: $O(nm)$
\end{itemize}

Query implementation:
\begin{itemize}
    \item For inputs $i$ and $j$, find each in the hash map, which will allow
        you to find the corresponding leaves in the prefix tree. $O(1)$
    \item Calculate $LCA$ of the two nodes. $O(1)$.
    \item Total runtime: $O(1)$
\end{itemize}

%Question 2
\item Given a rooted tree $T$, define $T(v)$ to be the set of leaves in $T$
    below $v$. For a set of nodes $V'$, define $LCA_T(V')$ to be the shallowest
    node $u$ in $T$ such that $u = LCA(w,x)$, for $w$,$x \in V'$

    Given two rooted trees, $S$ and $T$ with the same leaf labels, define, for
    each node $v$ in $S$, $M(v) = LCA_T(S(v))$.

    Give an algorithm to compute $M$.

    Implementation of $M$:
    \begin{itemize}
        \item Preprocess tree $T$ for $LCA$.
        \item Let $S$ have $n$ nodes. Let ${l}$ be the set of leaves below $v$
            in $S$. Let $T$ have $k$ nodes.
        \item Compute ${l}$ by doing an in order traversal of the tree with $v$
            as a root. $O(n)$.
        \item Find the locations of each node in ${l}$ in $T$. $O(k)$
        \item Recursively do $LCA$ on ${l}$ in $T$ by finding the $LCA$ of each
            pair of nodes until one common $LCA$ is found. $log(|{l}|)$ iterations
            of $LCA$ will be done. $O(log(|{l}|)$
        \item Total runtime: $n + k + log({|l|})$
        \item Since the number of nodes is a tree is greater than or equal to
            the number of leaves it has, we can say that $|{l}| \leq k$
        \item Therefore, total run time is $O(n+k)$
    \end{itemize}


\item The Set Cover Problem is defined as follows:

\item Let $T = (V,E)$ be an edge weighted tree such that $e \in E$ has minimal
    weight.  Let $T_1$ and $T_2$ be the trees derived from $T$ by removing $e$.
    Then we define a cartesian tree of $T$ to be a binary tree such that $e$ is
    the root, and the left and right children of $e$ are the cartesian tress of
    $T_1$ and $T_2$ respectively. If either $T_1$ or $T_2$ are singleton nodes,
    then their cartesian trees are empty.

    Give and algorithm for finding the cartesian tree of a tree. Give an
    analysis of its running time. The faster the algorithm, the better your
    grade.

    Implementation of algorithm A:
    \begin{itemize}
        \item If $T$.root has no children: done
        \item Do in order traversal to find $e$ and keep track of the deeper
            node that $e$ touches as the root of $T_2$. $O(n)$
        \item $T$.root = $e$. $O(1)$
        \item Go up the tree to find the root of $T_1$. $O(n)$
        \item $T$.root.left = $A(T_1)$
        \item $T$.root.right = $A(T_2)$
    \end{itemize}

    Run time analysis:
    \begin{itemize}
        \item $T(n) = 2T(\frac{n}{2}) + 2n$
        \item By Master Theorem: $T(n) is \Theta(nlogn)$
    \end{itemize}

\item Give a lower bound of $\Omega(nlogn)$ for constructing the cartesian tree
    of a tree.

    \begin{itemize}
        \item $T(1) = 0$
        \item $T(n) = 2T(\frac{n}{2}) + 2n$
        \item $a = 2$
        \item $b = 2$
        \item $c = 0$
        \item $d = 2$
        \item $f(n) = n$
        \item $a = 2 = f(b)$
        \item $\Theta(n^{\log 2} \log n)$
        \item $\Theta(nlogn)$
    \end{itemize}

\end{enumerate}

\end{document}
