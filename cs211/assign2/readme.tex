\documentclass[letterpaper,12pt]{article}

\usepackage[margin=1in]{geometry}

\begin{document}

\noindent Joshua Matthews

\noindent Computer Architecture

\noindent dataConvertor

\section{Design}
The program takes three arguments, and assuming everything was entered correctly, converts the input value to thw selected type of output.\\
If conversion to the same format as the input is requested, then the input is simply printed as is and the program exits.\\
However, if conversion between two different types is requested, the input value is first converted to a binary representation, and then to the requested output format.

\section{Efficiency}
During the conversion, the data may be itterated over several times. However, the number of itterations is always a specific ammount,
and not related to the lenght of the input. The efficiency therefore simplifies to O(n), where n is the length of the input data.

\end{document}
