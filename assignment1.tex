\documentclass[12pt]{article}
\usepackage[margin=0.75in]{geometry}
\usepackage{indentfirst}
\usepackage{listings}
\usepackage{amssymb}

\title{\bf Algorithms Homework 1}
\author{Kaitlin Poskaitis, Joshua Matthews, Russ Frank, and Matt Robinson}
\date{}

\begin{document}

\maketitle

\begin{enumerate}
%Question 1
\item Prove: A binary tree with $n$ nodes has a depth at least $\lfloor
log(n) \rfloor$ and at most $n - 1$.

Proof:

\begin{itemize}

    \item Assume the tree is full. Each level has $2^i$ nodes, where $i$ is the
        depth at that level.

    \item The total number of nodes in the tree is then
        $\displaystyle\sum\limits_{i=0}^d 2^i$

    \item This simplifies to $2^{d+1}-1$

    \item Since this is the max number of for depth $d$, we can say that $n \leq
        2^{d+1}-1$. Then by algebra:

    \item $n+1 \leq 2^{d+1}$

    \item $log(n+1) \leq (d+1)log2$

    \item $log(n+1) \leq d+1$

    \item $log(n+1) - 1 \leq d$

    % FIXME: and then a miracle happens...

    \item $\lfloor log(n) \rfloor \leq d$

\end{itemize}

For the maximum depth, it is impossible to have a depth greater than the number
of nodes, so the greatest possible depth is $n-1$, if such a configuration
exists. This type of tree can be trivially shown to exist by lining all the
nodes up down one side of the tree, resulting in a tree with $n-1$ depth.

%Question 2
\item Prove that $log(n!) \in \Theta(nlogn)$

%Question 3
\item Prove by induction on $k$.
\begin{itemize}
\item  $\displaystyle\sum\limits_{i=1}^k i(i+1) $ $= \frac{1}{3}k(k+1)(k+2)$\\

Base Case: $k=1$
$1(1+1) = 2$\\
$\frac{1}{3}(1)(1+1)(1+2) = 2$ So base case works\\

Assume everything up to and including $k$ works, prove $k+1$ works.\\
Need to prove: $\displaystyle\sum\limits_{i=1}^k i(i+1)$ $+(k+1)(k+2) =
\frac{1}{3}(k+1)(k+2)(k+3)$\\
$\displaystyle\sum\limits_{i=1}^k i(i+1)$ $+(k+1)(k+2) =$\\
$\frac{1}{3}k(k+1)(k+2)+(k+1)(k+2)$ through induction\\
$= (\frac{1}{3}k+1)(k+1)(k+2)$\\
$= \frac{1}{3}(k+3)(k+1)(k+2)$\\
$= \frac{1}{3}(k+1)(k+2)(k+3)$\\
QED
\item $\displaystyle\sum\limits_{i=0}^k i2^i $ $= (k-1)2^{k+1}+2$ \\

  \textbf{Base case.} \\ 
    $0 * 2^0 = 0$. \\
    $(0 - 1)2^{0 + 1} + 2 = 0$. \\ \\
  \textbf{Inductive step.}
    $\sum_{i=0}^k i2^i + (k + 1)2^{k+1} \stackrel{?}{=} k2^{k+2} + 2$ \\
    $(k - 1)2^{k+1} + 2 + (k + 1)2^{k+1}$ \\
    $((k + 1) + (k - 1))2^{k+1} + 2$ \\
    $(2k)2^{k+1}+2)$ \\
    $= k2^{k+2} + 2$, which is what we were trying to show. $\blacksquare$

\item $\displaystyle\sum\limits_{i=0}^k \frac{i}{2^i} $ $= 2-\frac{k+2}{2^k}$ \\

  \textbf{Base case.} \\
    $\frac{0}{2^0} = 0$ \\
    $2 - \frac{0 + 2}{2^0}$ \\
    $2 - 2 = 0$.\\

  \textbf{Inductive step.} \\
    $\displaystyle \sum_{i=0}^k \frac{i}{2^i} + \frac{k+1}{2^{k+1}} \stackrel{?}{=} 2 - \frac{k+3}{2^{k + 1}}$ \\
    $\displaystyle 2 - \frac{k+2}{2^k} + \frac{k+1}{2^{k+1}}$\\
    $\displaystyle 2 - \frac{2k+4}{2^{k+1}} + \frac{k+1}{2^{k+1}}$\\
    $\displaystyle 2 - \frac{k+3}{2^{k+1}} = 2 - \frac{k+3}{2^{k+1}}$, which is what we were trying to show. $\blacksquare$

\end{itemize}


%Question 4
\item Place the following functions into increasing asymptotic order. If two
or more of the functions are of the same asymptotic order, then indicate this.
Prove the correctness of your ordering.
Functions: $4n, n^2, nlogn, nlnn, logn, e^n$

$logn < 4n < nlogn = nlnn < n^2 < e^n$


$$\lim_{n\to\infty} \frac{logn}{4n} = \lim_{n\to\infty} \frac{\frac{1}{ln2n}}{4} = 0$$

$$\lim_{n\to\infty} \frac{4n}{nlogn} = \frac{4}{logn} = 0$$

$$\lim_{n\to\infty} \frac{nlogn}{nlnn} = \frac{logn}{lnn} =
\frac{\frac{lnn}{ln2}}{lnn} = \frac{1}{ln2}$$

$$\lim_{n\to\infty} \frac{nlnn}{n^2} = \frac{lnn}{n} = \lim_{n\to\infty}
\frac{\frac{1}{n}}{1} = 0$$

$$\lim_{n\to\infty} \frac{n^2}{e^n} = \lim_{n\to\infty} \frac{2n}{e^n} = 
\lim_{n\to\infty} \frac{2}{e^n} = 0$$


%Question 5
\item Let $G = (V,E,W)$ be a weighted graph such that no two different edges
in E have the same weight. Prove that $G$ has a unique unrooted minimal spanning
tree.\\
Proof:\\
Assume $G$ has at least two unrooted minimal spanning trees.\\
Let two of those trees be tress $A$ and $B$, which differ by one
edge.

Add the edge that is not in $A$ but is in $B$ to $A$.

$A$ now contains a cycle, so remove an edge. There must be one
correct choice as the edge weights are unique.

$A$ is now an unrooted minimal spanning tree that is either
smaller than the original $A$ or smaller than $B$. This is a contradiction.

Therefore, the assumption was wrong, so $G$ can only have one
unrooted minimal spanning tree.


%Question 6
\item Using the method $HamC(G)$, which returns Yes if $G$ contains a
Hamiltonian Cycle, and No otherwise, construct an algorithm to compute
$HamP(G,u,v)$.

\begin{lstlisting}
HamP(G,u,v)
    if u and v are neighbors:
        return HamC(G)
    Create edge between u and v
    if HamC(G) returns Yes:
        neighbors = v.neighbors
        remove v from graph
        for n in neighbors
            assign v to n
            if HamP(G,u,v) returns Yes:
                return Yes
    return No

\end{lstlisting}

\end{enumerate}

\end{document}
