\documentclass[12pt]{article}
\usepackage[margin=0.75in]{geometry}
\usepackage{indentfirst}
\usepackage{listings}
\usepackage{amssymb}

\title{\bf Algorithms Homework 2}
\author{Kaitlin Poskaitis, Joshua Matthews, Russ Frank, and Matt Robinson}
\date{}

\begin{document}

\maketitle

\begin{enumerate}
%Question 1
\item Suppose that you are given an $k-sorted$ array, in which no element is
    farther than $k$ positions away from its final (sorted) position. Give an
    algorithm which will sort such an array. Prove its correctness. Analyse its
    running time. Note: your algorithm should run faster than $\Theta(nlogn)$,
    that is it should take advantage of the fact that the array is almost
    sorted.


\textbf{Algorithm:}

\begin{enumerate}
    \item Given a $k-sorted$ list of $n$ elements, split the list into
        $\frac{n}{k}$ lists of $k$ elements.

    \item Perform a merge sort on each of the sub-lists. \hfill $O(\frac{n}{k} *
        klog(k)) = O(nlogk)$

    \item At this point each element will still be no greater than $k$ away from
        its sorted position.

    \item As in mergesort, merge all of these sorted sub-lists back into a
        single list. Each of these merges can move each element at most twice as
        far from it's initial position, but each successive merge will cover
        twice as many elements and eventually place each element in it's correct
        position. \hfill $O(n)$

    \item Final efficiency is $O(nlogk) + O(n) = O(nlogk)$


\end{enumerate}


%Question 2
\item Consider once again the $k-sorted$ array of the previous problem. Show
    that any comparison based algorithm for sorting an almost sorted array makes
    $\Omega(nlogk)$ comparisons.

%Question 3
\item Prove that if an $n$ node graph has two of the following properties, it has the
third:
\begin{itemize}
    \item It has $n-1$ edges.
    \item It is connected
    \item It is acyclic
\end{itemize}

%Question 4
\item Define a subtree of be any connected subgraph of a tree.
    \begin{itemize}
        \item Prove that the number of subtrees of a complete binary tree is not
            polynomial in the number of nodes.
        \item Give an example of a class of trees ${T_n}$ where the number of
            subtrees is a polynomial in the number of nodes.
    \end{itemize}

%Question 5
\item Let $F_i$ be the $i$th Fibonacci number. Show that $F_{n+2} = 1 +
    \sum_{i=0}^n F_i$.

%Question 6
\item Show that if you have a polynomial time algorithm for Hamiltonian Path,
    that you have a polynomial time algorithm for sorting.

    Proof:\\
    Assume there is a polynomial time algorithm for Hamiltonian Path.\\
    Merge sort is a polynomial sorting algorithm.\\
    QED

%Question 7
\item The Bounded Degree Spanning Tree (BDST) problem is the following:\\
    Input: Graph $G$ and integer $k$.\\
    Output: Yes, if $G$ has a spanning tree where every node has a degree of at
    most $k$, No, otherwise.\\
    Suppose there is no polynomial time algorithm for Hamilonian Path. Show that
    there is no polynomial time algorithm for BDST.

\end{enumerate}

\end{document}
